% Atur variabel berikut sesuai namanya

% nama
\newcommand{\name}{Jabalnur, S.T.}
\newcommand{\authorname}{Jabalnur, S.T.}
\newcommand{\nickname}{Jibi}
\newcommand{\advisor}{Dr. Agus Budi Raharjo, S.Kom, M.Kom., Ph.D.}
\newcommand{\examinerone}{None, S.T., M.Sc}
\newcommand{\examinertwo}{None, S.T., M.Sc}
\newcommand{\examinerthree}{None, ST., MT}
\newcommand{\headofdepartment}{Prof. Dr. Eng. Chastine Fatichah, S.Kom., M.Kom.}

% identitas
\newcommand{\nrp}{5025201241}
\newcommand{\advisornip}{1990202011022}
\newcommand{\examineronenip}{None}
\newcommand{\examinertwonip}{None}
\newcommand{\examinerthreenip}{None}
\newcommand{\headofdepartmentnip}{197512202001122002}

% judul
\newcommand{\tatitle}{\emph{INFLUENCE DEBUGGING AND TEST INPUT GENERATION}}
\newcommand{\engtatitle}{\emph{INFLUENCE DEBUGGING AND TEST INPUT GENERATION}}

% tempat
\newcommand{\place}{Surabaya}

% jurusan
\newcommand{\studyprogram}{Teknik Informatika}
\newcommand{\engstudyprogram}{Informatics Engineering}

% fakultas
\newcommand{\faculty}{Teknologi Elektro dan Informatika Cerdas}
\newcommand{\engfaculty}{Intelligent Electrical and Informatics Technology}

% singkatan fakultas
\newcommand{\facultyshort}{FTEIC}
\newcommand{\engfacultyshort}{F-ELECTICS}

% departemen
\newcommand{\department}{Teknik Informatika}
\newcommand{\engdepartment}{Informatics Engineering}

% institute
\newcommand{\institute}{Institut Teknologi Sepuluh Nopember}
\newcommand{\enginstitute}{Sepuluh Nopember Institute of Technology}

% kode mata kuliah
\newcommand{\coursecode}{EF234704}

% email
\newcommand{\email}{jabalnur.it@gmail.com}

\begin{center}
  \large\textbf{ABSTRACT}
\end{center}

\vspace{2ex}

\begingroup
% Menghilangkan padding
\setlength{\tabcolsep}{0pt}

\noindent
\begin{tabularx}{\textwidth}{l >{\centering}m{3em} X}
  \emph{Name}     & : & \name{}         \\

  \emph{Title}    & : & \engtatitle{}   \\

  \emph{Advisor}    & : & \advisor{}   \\

\end{tabularx}
\endgroup

% Ubah paragraf berikut dengan abstrak dari tugas akhir dalam Bahasa Inggris
% \emph{Large-scale data processing frameworks like Apache Spark have enabled developers to process petabytes of data with Data-Intensive Scalable Computing (DISC) applications. Like any other software, faults in DISC applications are common. These faults can arise due to unclean input datasets or application implementation, requiring developers to identify the root cause among billions of input data. The process of Data Provenance and automated debugging requires instrumenting DISC applications or performing search-based fault isolation, which consumes significant resources.}

% \emph{This study aims to demonstrate the potential use of language models to generate minimal input datasets capable of reproducing the same faults observed in the original large-scale input data. We introduce FISUM (Fault-inducing Inputs Summarization), which trains a model on historically faulty inputs to generate new minimal faulty rows. FISUM leverages the Data Provenance Engine built for Apache Spark to recover fault-inducing inputs. It then trains a lightweight language model, DistilGPT, on these failure-inducing inputs to generate minimal fault-revealing inputs.}

% \emph{In experiments with existing faulty DISC applications, even with a lightweight model, FISUM effectively summarizes fault-inducing input, reducing the original input from the data provenance tool by 99\%, while still inducing the same fault. FISUM generates faulty inputs in under 50 milliseconds per row, on average.}

\emph{Large-scale data processing frameworks like Apache Spark have enabled developers to process petabytes of data with Data-Intensive Scalable Computing (DISC) applications. Like other software, errors in DISC applications are common. These errors can arise due to unclean input datasets or application implementation issues, requiring developers to identify the root cause among billions of input data. Data Provenance and automatic debugging processes require DISC application instrumentation or resource-intensive search-based fault isolation. This research aims to demonstrate the potential use of language models to generate minimal input datasets that can reproduce the same errors observed in the original large-scale input data. We created FISUM (Fault-inducing Inputs Summarization), which trains a model with historically faulty input to produce minimal new faulty rows with similar characteristics to the original data.}

\emph{FISUM uses Titian, a Data Provenance Engine built for Apache Spark, to recover inputs that caused errors. A Data Provenance Engine is a system that tracks the origin, history, and journey of data in a computing system to help identify the source of errors and understand data transformations from the original source to its final form. Titian is then embedded into DISC applications with problematic outputs to find Faulty Input, i.e., data rows in the dataset that cause problematic outputs in the DISC application. A lightweight language model, HuggingFace's DistilGPT2, is then trained using the initial DISC application dataset to familiarize the model with the application data. This model is further trained with Faulty Input from Titian to understand the characteristics of the Faulty Input. The trained DistilGPT2 model can then generate new Faulty Input that is much smaller in size compared to the original while retaining similar error characteristics to the initial Faulty Input within a short time.}

\emph{Experiments were conducted by testing 16 Benchmark Applications, which are DISC applications, to measure FISUM's accuracy in finding Faulty Input in the application dataset, the accuracy of errors in new Faulty Input, and the Mean Square Error (MSE) of characteristics between the initial and new Faulty Input. FISUM effectively identifies Faulty Input in DISC applications with 100\% accuracy, summarizes Faulty Input with 99.4\% accuracy in less than 0.082 seconds per row on average, and presents similarity in characteristics between the initial and new Faulty Input with an MSE of 0.262.}


% Ubah kata-kata berikut dengan kata kunci dari tugas akhir dalam Bahasa Inggris
\emph{Keywords}: \emph{Debugging}, \emph{Generative AI}, \emph{Faults}, \emph{Language Models}.

\begin{center}
  \large\textbf{ABSTRACT}
\end{center}

\vspace{2ex}

\begingroup
% Menghilangkan padding
\setlength{\tabcolsep}{0pt}

\noindent
\begin{tabularx}{\textwidth}{l >{\centering}m{3em} X}
  \emph{Name}     & : & \name{}         \\

  \emph{Title}    & : & \engtatitle{}   \\

  \emph{Advisor}    & : & \advisor{}   \\

  \\
\end{tabularx}
\endgroup

% Ubah paragraf berikut dengan abstrak dari tugas akhir dalam Bahasa Inggris
\emph{Large-scale data processing frameworks like Apache Spark have enabled developers to process petabytes of data with Data-Intensive Scalable Computing (DISC) applications. Like any other software, faults in DISC applications are common. These faults can arise due to unclean input datasets or application implementation, requiring developers to identify the root cause among billions of input data. The process of Data Provenance and automated debugging requires instrumenting DISC applications or performing search-based fault isolation, which consumes significant resources.}

\emph{This study aims to demonstrate the potential use of language models to generate minimal input datasets capable of reproducing the same faults observed in the original large-scale input data. We introduce FISUM (Fault-inducing Inputs Summarization), which trains a model on historically faulty inputs to generate new minimal faulty rows. FISUM leverages the Data Provenance Engine built for Apache Spark to recover fault-inducing inputs. It then trains a lightweight language model, DistilGPT, on these failure-inducing inputs to generate minimal fault-revealing inputs.}

\emph{In experiments with existing faulty DISC applications, even with a lightweight model, FISUM effectively summarizes fault-inducing input, reducing the original input from the data provenance tool by 99\%, while still inducing the same fault. FISUM generates faulty inputs in under 50 milliseconds per row, on average.}

% Ubah kata-kata berikut dengan kata kunci dari tugas akhir dalam Bahasa Inggris
\emph{Keywords}: \emph{Debugging}, \emph{Generative AI}, \emph{Faults}, \emph{Language Models}.

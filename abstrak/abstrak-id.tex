\chapter*{ABSTRAK}
\addcontentsline{toc}{chapter}{ABSTRAK}

\vspace{2ex}

\begingroup
% Menghilangkan padding
\setlength{\tabcolsep}{0pt}

\noindent
\begin{tabularx}{\textwidth}{l >{\centering}m{2em} X}
  Nama Mahasiswa    & : & \name{}         \\

  Judul Tugas Akhir & : & \tatitle{}      \\

  Pembimbing        & : & 1. \advisor{}   \\
  \\
\end{tabularx}
\endgroup

% Ubah paragraf berikut dengan abstrak dari tugas akhir
Framework pemrosesan data skala besar seperti \textit{Apache Spark} telah memungkinkan \textit{developer} untuk memproses petabyte data dengan aplikasi \textit{Data-Intensive Scalable Computing (DISC)}. Seperti halnya perangkat lunak lainnya, kesalahan dalam aplikasi \textit{DISC} adalah hal yang umum. Kesalahan tersebut dapat timbul karena dataset input yang tidak bersih atau implementasi aplikasi, yang membuat \textit{developer} harus mengidentifikasi akar penyebab di antara miliaran data input. Proses \textit{Data Provenance} dan \textit{debugging} otomatis memerlukan instrumentasi aplikasi \textit{DISC} atau melakukan \textit{search-based fault isolation} yang memakan banyak sumber daya.

Penelitian ini bertujuan untuk menunjukkan potensi penggunaan \textit{language model} untuk menghasilkan dataset input minimal namun dapat mereproduksi kesalahan yang sama seperti yang diamati pada data masukan skala besar aslinya. Kami membuat \textit{FISUM} (\textit{Fault-inducing Inputs Summarization}), yang melatih model dengan \textit{faulty input} secara historis untuk menghasilkan baris salah baru yang minimal. \textit{FISUM} menggunakan \textit{Data Provenance Engine} yang dibangun untuk \textit{Apache Spark}, untuk memulihkan input yang menyebabkan kesalahan. Kemudian melatih sebuah \textit{lightweight language model}, \textit{DistilGPT}, menggunakan input yang menyebabkan kegagalan ini untuk menghasilkan input minimal yang mengungkapkan kesalahan yang sama.

Pada percobaan dengan aplikasi \textit{DISC} bermasalah yang telah ada, bahkan dengan model ringan, \textit{FISUM} secara efektif dapat merangkum input yang menyebabkan kesalahan, mengurangi input asli dari alat \textit{data provenance} sebesar 99\%, namun masih dapat menyebabkan kesalahan yang sama. \textit{FISUM} menghasilkan \textit{faulty input} dalam waktu kurang dari 50 milidetik per baris, rata-rata.

% Ubah kata-kata berikut dengan kata kunci dari tugas akhir
Kata Kunci: \emph{Debugging}, \emph{Generative AI}, \emph{Faults}, \emph{Language Models}.

\chapter{PENDAHULUAN}\label{chap:pendahuluan}

% Ubah bagian-bagian berikut dengan isi dari pendahuluan

\section{Latar Belakang}\label{sec:latarbelakang}

Analisis data modern memerlukan penanganan data berskala 
terabyte yang tersebar di berbagai mesin. Untuk membuat 
ini dapat dilakukan, kerangka kerja 
\emph{Data Intensive Scalable Computing (DISC)} seperti 
Apache Spark~\cite{zaharia2010,spark} dan 
Google MapReduce~\cite{dean2008} memungkinkan analis 
data untuk membuat aplikasi terdistribusi. Volume besar 
data input untuk aplikasi DISC, ditambah dengan sifat 
terdistribusi dari program, membuat debugging menjadi 
sangat menantang.
Bayangkan sebuah program yang menghitung rata-rata 
curah hujan per tahun di Amerika Serikat. Ini 
memerlukan penghitungan rata-rata dari puluhan 
juta nilai yang dikumpulkan dari sensor yang 
tersebar di seluruh negeri. Setelah menjalankan 
program, analis memperhatikan bahwa nilai rata-rata 
yang dihasilkan sangat tinggi dan mencurigakan. 
Bagaimana programmer akan mendiagnosis penyebab 
kesalahan ini?

Beberapa penelitian dalam debugging program telah 
berfokus pada pengurangan ukuran input yang 
menyebabkan kesalahan~\cite{zeller2002,misherghi2006,kirschner2020,clause2009}.
Banyak dari teknik ini 
adalah varian dari algoritma delta debugging, 
yang bekerja dengan menjalankan program berulang 
kali dengan segmen-segmen berbeda dari input yang 
menyebabkan kesalahan hingga algoritma tersebut 
menemukan ukuran input minimal. Teknik pengurangan 
input tradisional ini, meskipun telah dicoba~\cite{gulzar2018}, tidak 
dapat diterapkan pada aplikasi DISC karena memerlukan 
eksekusi berulang yang lambat dan memakan banyak 
sumber daya.

Dalam penelitian ini, kami mengusulkan teknik berbasis 
\emph{large language model} (LLM) yang efisien untuk 
merangkum data yang bermasalah, yang mungkin mencakup 
jutaan baris, menjadi hanya beberapa baris yang tetap 
dapat mereproduksi kesalahan. Kami mewujudkan ide ini 
dalam alat yang disebut FISUM, sebuah sistem 
\emph{Fault-inducing Inputs Summarization}. 
Inti dari teknik kami adalah bahwa LLM generatif 
modern dapat dilatih untuk menangkap pola input 
yang memicu kesalahan dari data besar. Hasil dari 
pelatihan model ini kemudian dapat digunakan untuk 
menghasilkan input baru yang menyebabkan pola 
kesalahan yang sama.

Teknik kami memerlukan proses \emph{pre-training} 
dari model GPT versi ringan, distilGPT, pada seluruh 
data input. Langkah \emph{unsupervised} ini 
memungkinkan model untuk mempelajari struktur 
dasar dan pola dari data tersebut. Kami kemudian 
menggunakan Titian~\cite{interlandi2015} untuk mengambil subset besar 
dari input yang dianggap mencurigakan terkait 
dengan eksekusi yang salah. distilGPT kemudian 
dilatih ulang pada data mencurigakan ini untuk 
membiasakannya dalam menghasilkan input yang 
menyebabkan kesalahan. Akhirnya, distilGPT diarahkan 
untuk menghasilkan data yang dapat mereproduksi 
kesalahan tersebut.
Untuk mengevaluasi teknik kami, kami menggunakan satu 
set program \emph{benchmark} dari penelitian sebelumnya 
tentang \emph{debugging} dan pengujian aplikasi 
DISC~\cite{gulzar2019,humayun2023,zhang2021}. 

\section{Rumusan Permasalahan}
\label{sec:permasalahan}

Berdasarkan latar belakang di atas, maka dapat ditarik rumusan masalah sebagai berikut:

\begin{enumerate}[nolistsep]

   \item Apakah input yang bermasalah yang dihasilkan oleh FISUM masih mempertahankan kemampuan deteksi kesalahan?

   \item Berapa banyak persentase pengurangan input bermasalah yang dapat dihasilkan oleh FISUM dari dataset yang diberikan?

   \item Berapa lama waktu yang dibutuhkan untuk menghasilkan input bermasalah oleh FISUM?
   
   \item Berapa lama waktu yang dibutuhkan untuk melatih tiap model dengan FISUM?

\end{enumerate}

\section{Batasan Masalah}
\label{sec:batasanmasalah}

Batasan masalah penelitian ini adalah sebagai berikut:

\begin{enumerate}[nolistsep]

  \item Implementasi algoritma menggunakan bahasa Python dan Scala.

  \item Pembuatan sistem hanya berfokus pada proses \emph{debugging} dan \emph{generate new faulty input} bukan pada aplikasi DISC yang test inputnya akan didebug.

  \item Proses \emph{debugging} dan \emph{generate new faulty input} hanya berfokus pada 16 \emph{benchmark program} yang berjalan di aplikasi DISC. 

  \item Test input berupa \emph{text} yang menyatukan beberapa kolom yang dibutuhkan

\end{enumerate}

\section{Tujuan}
\label{sec:Tujuan}

Tujuan penelitian ini adalah sebagai berikut:

\begin{enumerate}[nolistsep]

  \item Mengevaluasi kemampuan deteksi kesalahan dari input bermasalah yang dihasilkan oleh FISUM.
  \item Mengukur persentase pengurangan input bermasalah yang dihasilkan oleh FISUM dari dataset yang diberikan.
  \item Mengukur waktu yang dibutuhkan untuk menghasilkan input bermasalah oleh FISUM.
  \item Mengukur waktu yang dibutuhkan untuk melatih tiap model dengan FISUM.

\end{enumerate}

\section{Manfaat}
\label{sec:Manfaat}

Manfaat penelitian ini adalah sebagai berikut:

\begin{enumerate}[nolistsep]
  \item Membantu \ meningkatkan \ kualitas \  perangkat \  lunak \ yang \  berjalan \ pada \ platform \emph{Data-Intensive Scalable Computing (DISC)},\  seperti \ Apache \ Hadoop, \ Apache Spark, dan Apache Flink.
  \item Membantu menghemat waktu dan upaya yang diperlukan untuk mengidentifikasi dan mengatasi masalah dalam aplikasi DISC.
  \item Meningkatkan produktivitas pengembang perangkat lunak.
  \item Membantu penilitian yang terkait dengan aplikasi DISC.

\end{enumerate}

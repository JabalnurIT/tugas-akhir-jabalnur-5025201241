\chapter{HASIL DAN PEMBAHASAN}
\label{chap:hasildanpembahasan}

Pada penelitian ini dipaparkan \lipsum[1][1-5]

\section{Hasil Eksperimen}
\label{sec:hasilpengujian}

\subsection{Lingkungan Pengujian}
\label{subsec:lingkunganpengujian}

\lipsum[1]

\subsection{Skenario Uji Coba}
\label{subsec:skenarioujicoba}

\lipsum[2]

% item
\begin{itemize}
  \item \textbf{Fault Detection Capability} \\
        \lipsum[3]
  \item \textbf{Data Reduction} \\
        \lipsum[4]
  \item \textbf{Data Generation Time} \\
        \lipsum[5]
  \item \textbf{Model Training Time} \\
        \lipsum[6]
  \item \textbf{Data Generation Time in Cloud} \\
        \lipsum[7]
  \item \textbf{Model Training Time in Cloud} \\
        \lipsum[8]
\end{itemize}

\section{Pembahasan}
\label{sec:pembahasan}

Dari pengujian yang \lipsum[1]

% Contoh pembuatan tabel
\begin{longtable}{|c|c|c|}
  \caption{Hasil Pengukuran Energi dan Kecepatan}
  \label{tb:EnergiKecepatan}                                   \\
  \hline
  \rowcolor[HTML]{C0C0C0}
  \textbf{Energi} & \textbf{Jarak Tempuh} & \textbf{Kecepatan} \\
  \hline
  10 J            & 1000 M                & 200 M/s            \\
  20 J            & 2000 M                & 400 M/s            \\
  30 J            & 4000 M                & 800 M/s            \\
  40 J            & 8000 M                & 1600 M/s           \\
  \hline
\end{longtable}

\lipsum[2]

\subsection{Analisis}
\label{subsec:analisis}

\lipsum[3]

\subsection{Evaluasi}
\label{subsec:evaluasi}

\lipsum[4]

% item
\begin{itemize}
  \item \textbf{Fault Detection Capability} \\
        \lipsum[5]
  \item \textbf{Data Reduction} \\
        \lipsum[6]
  \item \textbf{Data Generation Time} \\
        \lipsum[7]
  \item \textbf{Model Training Time} \\
        \lipsum[8]
  \item \textbf{Data Generation Time in Cloud} \\
        \lipsum[9]
  \item \textbf{Model Training Time in Cloud} \\
        \lipsum[10]
\end{itemize}

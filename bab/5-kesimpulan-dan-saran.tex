\chapter{KESIMPULAN DAN SARAN}
\label{chap:kesimpulandansaran}

% Ubah bagian-bagian berikut dengan isi dari penutup

\section{Kesimpulan}
\label{sec:kesimpulan}

Dalam penelitian ini, kami memperkenalkan FISUM, 
sebuah alat peringkasan input yang menyebabkan 
kesalahan yang dapat mereproduksi kesalahan aplikasi 
DISC dengan baris data yang jauh lebih sedikit dibandingkan 
dengan teknologi saat ini. Dari hasil penelitian ini,
kami dapat menyimpulkan beberapa hal sebagai berikut:
% make enumerate nolistsep
% - Kemampuan Deteksi Kesalahan: Pada 14 dari 16 program benchmark, FISUM mampu mereproduksi kesalahan asli dalam ≤ 1% dari ukuran input yang menyebabkan kesalahan asli yang dikembalikan oleh Titian. Ini menunjukkan bahwa input bermasalah yang dihasilkan oleh FISUM masih mempertahankan kemampuan deteksi kesalahan yang tinggi.

% - Pengurangan Input Bermasalah: FISUM berhasil merangkum input yang menyebabkan kesalahan menjadi 1% dari input asli yang menyebabkan kesalahan pada sebagian besar program benchmark. Ini menunjukkan bahwa FISUM mampu menghasilkan persentase pengurangan input bermasalah yang signifikan dari dataset yang diberikan.

% - Waktu Pembuatan Data: Hasil eksperimen menunjukkan bahwa FISUM dapat menghasilkan ringkasan input bermasalah dengan waktu yang efisien. Misalnya, dalam program Movie Rating, FISUM membutuhkan hanya 60 milidetik untuk menghasilkan satu baris, menghasilkan total 0,3 detik untuk menghasilkan ringkasan 1% dari input yang menyebabkan kesalahan.

% - Waktu Pelatihan Model: Waktu pelatihan model menggunakan FISUM bervariasi tergantung pada program, dengan waktu terlama adalah 13,5 ribu detik pada Program 16, Word Count, dan waktu terpendek adalah 103 detik pada Program 2, Commute Type. Ini menunjukkan bahwa FISUM memerlukan waktu yang berbeda-beda untuk melatih tiap model, namun tetap dalam batas yang dapat diterima untuk debugging lokal yang interaktif.

\begin{enumerate}[nolistsep]

  \item \textbf{Kemampuan Deteksi Kesalahan:} 
  Pada 14 dari 16 program benchmark, 
  FISUM mampu mereproduksi dataset bermasalah
  dengan baris dataset sebanyak $\approx 1\%$ dari ukuran 
  dataset asli yang menyebabkan kesalahan yang
  dikembalikan oleh Titian. Ini menunjukkan bahwa input 
  bermasalah yang dihasilkan oleh FISUM masih 
  mempertahankan kemampuan deteksi kesalahan yang tinggi
  dengan tingkat akurasi 99\%.
  \item \textbf{Pengurangan Input Bermasalah:} 
  FISUM berhasil merangkum dataset yang menyebabkan 
  kesalahan menjadi 1\% dari dataset asli yang menyebabkan 
  kesalahan pada sebagian besar program \emph{benchmark}. 
  Ini menunjukkan bahwa FISUM mampu menghasilkan persentase 
  pengurangan input bermasalah yang signifikan dari 
  dataset yang diberikan.
  \item \textbf{Waktu Pembuatan Data:} 
  Hasil penelitian menunjukkan bahwa FISUM dapat 
  menghasilkan ringkasan input bermasalah dengan 
  waktu yang efisien. Misalnya, dalam program 
  \emph{Movie Rating}, FISUM membutuhkan hanya 
  60 milidetik untuk menghasilkan satu baris data, 
  dengan total 0,3 detik untuk menghasilkan dataset baru
  sebanyak 1\% dari input yang menyebabkan kesalahan
  dengan tetap mempertahankan retensi kesalahan.
  \item \textbf{Waktu Pelatihan Model:} 
  Waktu pelatihan model menggunakan FISUM bervariasi 
  tergantung pada program, dengan waktu terlama 
  adalah 13,5 ribu detik pada Program \#16, 
  \emph{Word Count}, dan waktu terpendek adalah 
  103 detik pada Program \#2, \emph{Commute Type}. 
  Ini menunjukkan bahwa FISUM memerlukan waktu yang 
  berbeda-beda untuk melatih tiap model, namun tetap 
  dalam batas yang dapat diterima untuk debugging 
  lokal yang interaktif.

\end{enumerate}


\section{Saran}
\label{chap:saran}

Untuk pengembangan lebih lanjut pada penelitian ini,
beberapa saran yang dapat dilakukan antara lain:

\begin{enumerate}[nolistsep]

    \item \textbf{Percobaan dengan Menggunakan GPU:}
    Melakukan percobaan menggunakan GPU dapat meningkatkan
    kecepatan pemrosesan dan pelatihan model FISUM. Dengan
    menggunakan GPU, kita dapat mengevaluasi sejauh mana
    peningkatan performa yang dapat dicapai dibandingkan
    dengan penggunaan laptop standar tanpa GPU.
    \item \textbf{Eksplorasi dengan Model Bahasa yang Lebih Modern:}
    Menggali penggunaan model bahasa yang lebih modern dan
    canggih dapat meningkatkan kecepatan dan ketepatan FISUM
    dalam merangkum input yang menyebabkan kesalahan. Namun,
    penting untuk tetap memperhatikan ukuran dari model
    tersebut agar tidak terlalu besar dan tetap efisien dalam
    penggunaannya pada data skala besar.

\end{enumerate}

\chapter{KESIMPULAN DAN SARAN}
\label{chap:kesimpulandansaran}

% Ubah bagian-bagian berikut dengan isi dari penutup

\section{Kesimpulan}
\label{sec:kesimpulan}

Berdasarkan hasil penelitian yang telah dilakukan, 
diperoleh kesimpulan sebagai berikut:
\begin{enumerate}[nolistsep]

  \item FISUM mampu mengidentifikasi \emph{faulty input} dari aplikasi DISC
  dengan cara melakukan \emph{provenance data} menggunakan Spark Titian dengan akurasi sebesar 100\%.
  \item FISUM mampu memproduksi \emph{faulty input} baru dari \emph{faulty input} yang telah teridentifikasi dalam waktu singkat
  dengan melakukan \emph{training} model DistilGPT2 dengan akurasi sebesar 99.4\% dalam waktu 0.082 detik per baris data.
  \item Kemiripan karakteristik antara \emph{faulty input} baru dan \emph{faulty input} yang telah teridentifikasi dengan Titian 
  dapat dievaluasi dengan menggunakan \emph{confusion matrix} yang menghasilkan nilai akurasi sebesar 99.4\%
  dan dengan menghitung nilai MSE yang menghasilkan nilai sebesar 0.262.

\end{enumerate}


\section{Saran}
\label{chap:saran}

Untuk pengembangan lebih lanjut pada penelitian ini,
beberapa saran yang dapat dilakukan antara lain:

\begin{enumerate}[nolistsep]

    \item Melakukan percobaan menggunakan GPU untuk dapat meningkatkan
    kecepatan pemrosesan dan pelatihan model FISUM. 
    \item Menggali penggunaan model bahasa yang lebih modern dan
    canggih untuk dapat mening-katkan kecepatan dan ketepatan FISUM
    dalam merangkum input yang menyebabkan kesalahan. 

\end{enumerate}
